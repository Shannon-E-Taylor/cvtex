%------------------------------------
% Dario Taraborelli
% Typesetting your academic CV in LaTeX
%
% URL: http://nitens.org/taraborelli/cvtex
% DISCLAIMER: This template is provided for free and without any guarantee 
% that it will correctly compile on your system if you have a non-standard  
% configuration.
% Some rights reserved: http://creativecommons.org/licenses/by-sa/3.0/
%------------------------------------

%!TEX TS-program = xelatex
%!TEX encoding = UTF-8 Unicode

\documentclass[10pt, a4paper]{article}
\usepackage{fontspec} 

% DOCUMENT LAYOUT
\usepackage{geometry} 
\geometry{a4paper, textwidth=5.5in, textheight=8.5in, marginparsep=7pt, marginparwidth=.6in}
\setlength\parindent{0in}

% FONTS
\usepackage[usenames,dvipsnames]{color}
\usepackage{xunicode}
\usepackage{xltxtra}

% ---- CUSTOM COMMANDS
\chardef\&="E050
\newcommand{\html}[1]{\href{#1}{\scriptsize\textsc{[html]}}}
\newcommand{\pdf}[1]{\href{#1}{\scriptsize\textsc{[pdf]}}}
\newcommand{\doi}[1]{\href{#1}{\scriptsize\textsc{[doi]}}}
% ---- MARGIN YEARS
\usepackage{marginnote}
\newcommand{\amper{}}{\chardef\amper="E0BD }
\newcommand{\years}[1]{\marginnote{\scriptsize #1}}
\renewcommand*{\raggedleftmarginnote}{}
\setlength{\marginparsep}{7pt}
\reversemarginpar

% HEADINGS
\usepackage{sectsty} 
\usepackage[normalem]{ulem} 
\sectionfont{\mdseries\upshape\Large}
\subsectionfont{\mdseries\scshape\normalsize} 
\subsubsectionfont{\mdseries\upshape\large} 

%make multicolumn bits for references 
\usepackage{multicol}





% PDF SETUP
% ---- FILL IN HERE THE DOC TITLE AND AUTHOR
\usepackage[driverfallback = dvipdfm, bookmarks, colorlinks, breaklinks, 
% ---- FILL IN HERE THE TITLE AND AUTHOR
	pdftitle={Shannon Taylor},
	pdfauthor={Shannon Taylor},
	pdfproducer={http://nitens.org/taraborelli/cvtex}
]{hyperref}  
\hypersetup{linkcolor=blue,citecolor=blue,filecolor=black,urlcolor=MidnightBlue} 
% DOCUMENT
\begin{document}
{\LARGE Shannon Taylor}\\[1cm]
Phone: \texttt{022-184-6925}\\
email: \href{mailto:shannon.elisa.taylor@gmail.com}{shannon.elisa.taylor@gmail.com}\\

%%\hrule
%\section*{Current position}
%\emph{Emeritus Professor}, Institute for Advanced Study, Princeton

%%\hrule
%\section*{Areas of specialization}
% Physics • Relativity theory

% %%\hrule
% \section*{Appointments held}
% \noindent
% \years{1903-1908}Swiss Patent Office, Bern\\
% \years{1908-1911}University of Bern\\
% \years{1911-1912}University of Zürich\\
% \years{1912-1914}Charles University of Prague\\
% \years{1914-1932}Prussian Academy of Sciences, Berlin\\
% \years{1920-1930}University of Leiden\\
% \years{1932-1955}Institute for Advanced Study, Princeton

%\hrule
\section*{Education}
\noindent
\years{2018}First Class Honours, BBioMedSci\\
Reproduction, Genetics, and Development, University of Otago\\
Thesis title: The role of \emph{Numb} in Honeybee Ovary Activation\\
Supervisor: Prof. Peter Dearden\\
\years{2015 - 2017}BBioMedSci, Molecular Basis of Health and Disease, University of Otago\\
GPA: 8.3/9 \\
\years{2010 - 2014}Westminster School, Adelaide, Australia \\
ATAR 97.45 \\


\section*{Publications and Poster Presentations}
\noindent
\years{2018}\textbf{Taylor, S.E.}, Tuffrey, J., and Dearden, P.K. (2018) ``\emph{Torso-like} is necessary for vitelline membrane integrity in \emph{Nasonia vitripennis}'', Euro Evo Devo 2018, Galway, Ireland. \\
\years{2017} Cridge, A. G., Lovegrove, M, Skelly, J. G., \textbf{Taylor, S. E.}, Petersen, G. E.L., Cameron, R. C., and Dearden, P. K. (2017), “The honeybee as a model insect for developmental genetics", \emph{Genesis} 55: 5, DOI: 10.1002/dvg.23019\\
\years{In prep.}\textbf{Taylor, S.E.}, Tuffrey, J., Lequeux, S and Dearden, P.K. ``The \emph{Drosophila} axis formation gene \emph{torso-like} functions to maintain the structure of the vitelline membrane in \emph{Nasonia vitripennis}''. Journal article in preparation. 

%\hrule
\section*{Scholarships and Awards}
\noindent
\years{2017 - 2018}Elizabeth Jean Trotter Scholarship in Biomedical Sciences \\
\years{2017}GSA Student Travel Award to attend the ``2017 Annual Conference of the Genetics Society of Australasia with the NZ Society for Biochemistry and Molecular Biology'' \\
\years{2015} Academic Excellence Toroa College \\
\years{2015} Certificate of Appreciation for Community Service Toroa College \\

\section*{Research Experience}
\years{2015-}Laboratory of Evolution and Development, Otago University \\ 
Molecular biology, genetics, and microscopy techniques to investigate the role of \emph{torso-like} in wasp development \\
Quantitative imaging to investigate the role of Notch signalling in honeybee ovary activation \\

\section*{Teaching Experience}
\years{2018}Department of Biochemistry, Otago University \\ 
Tutored groups of 15 second-year medical students in Genetics \\
Demonstrated Biochemistry laboratory techniques to first-year students. \\ 

\section*{Relevant skills} 
Investigating gene expression using \emph{in situ} hybridization, immunohistochemistry, hybridization chain reaction \\
Imaging using light microscopy, confocal microscopy, and some electron microscopy \\
Programming: beginner to intermediate python, R, git. \\



\section*{References}

\begin{multicols}{2}

\textbf{Prof. Peter Dearden}\\
Academic supervisor\\ 
Laboratory for Evolution and Development\\ 
Otago University\\
+64 3 479 7832\\
Email: peter.dearden@otago.ac.nz\\

\textbf{Sharleen Rae-Whitcombe}\\
Demonstrating supervisor\\ 
Department of Biochemistry\\
Otago University\\ 
+64 3 479 7083\\
Email: sharleen.rae-whitcombe@otago.ac.nz\\


\end{multicols}


\end{document}